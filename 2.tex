\documentclass[letterpaper, 10pt]{article}
\usepackage{amsmath, amsthm, amssymb}
\usepackage{mathtools}
\usepackage{enumitem}
\usepackage{listings}

\setlength{\parindent}{0pt}

\newtheorem{thm}{Theorem}
\numberwithin{thm}{section}
\newtheorem{lem}[thm]{Lemma}
\newtheorem{col}[thm]{Corollary}

\theoremstyle{definition}
\newtheorem{mydef}[thm]{Definition}

\begin{document}
\setcounter{section}{6}
\section{$r$-\uppercase{combinations}}
An \textit{r-combination of an n-set} $C(n, r)$ is a selection of $r$ elements from the set. Order
does not matter. $r$-combination is an $r$-element subset.
\begin{thm}
	\begin{equation*}
		P(n,r) = C(n,r) \times P(r,r)
	\end{equation*}
\end{thm}

\begin{col}
	\begin{equation}
		C(n,r) = \frac{n!}{r!\,(n-r)!}
	\end{equation}
\end{col}

\begin{col}
	\begin{equation*}
		C(n,r)=C(n,n-r)
	\end{equation*}
\end{col}

\emph{Binomial coefficient}
\begin{equation*}
	\binom{n}{r} = \frac{n!}{r!\,(n-r)!}
\end{equation*}

\begin{equation*}
	\binom{n}{r} = \binom{n}{n-r} 
\end{equation*}

\begin{thm}
	\begin{equation*}
		C(n,r)=C(n-1,r-1) + C(n-1,r)
	\end{equation*}
\end{thm}

\end{document}
